
% Default to the notebook output style

    


% Inherit from the specified cell style.




    
\documentclass[11pt, a4paper, landscape]{scrartcl}

    
    
    \usepackage[T1]{fontenc}
    % Nicer default font (+ math font) than Computer Modern for most use cases
    \usepackage{mathpazo}

    % Basic figure setup, for now with no caption control since it's done
    % automatically by Pandoc (which extracts ![](path) syntax from Markdown).
    \usepackage{graphicx}
    % We will generate all images so they have a width \maxwidth. This means
    % that they will get their normal width if they fit onto the page, but
    % are scaled down if they would overflow the margins.
    \makeatletter
    \def\maxwidth{\ifdim\Gin@nat@width>\linewidth\linewidth
    \else\Gin@nat@width\fi}
    \makeatother
    \let\Oldincludegraphics\includegraphics
    % Set max figure width to be 80% of text width, for now hardcoded.
    \renewcommand{\includegraphics}[1]{\Oldincludegraphics[width=.8\maxwidth]{#1}}
    % Ensure that by default, figures have no caption (until we provide a
    % proper Figure object with a Caption API and a way to capture that
    % in the conversion process - todo).
    \usepackage{caption}
    \DeclareCaptionLabelFormat{nolabel}{}
    \captionsetup{labelformat=nolabel}

    \usepackage{adjustbox} % Used to constrain images to a maximum size 
    \usepackage{xcolor} % Allow colors to be defined
    \usepackage{enumerate} % Needed for markdown enumerations to work
    \usepackage{geometry} % Used to adjust the document margins
    \usepackage{amsmath} % Equations
    \usepackage{amssymb} % Equations
    \usepackage{textcomp} % defines textquotesingle
    % Hack from http://tex.stackexchange.com/a/47451/13684:
    \AtBeginDocument{%
        \def\PYZsq{\textquotesingle}% Upright quotes in Pygmentized code
    }
    \usepackage{upquote} % Upright quotes for verbatim code
    \usepackage{eurosym} % defines \euro
    \usepackage[mathletters]{ucs} % Extended unicode (utf-8) support
    \usepackage[utf8x]{inputenc} % Allow utf-8 characters in the tex document
    \usepackage{fancyvrb} % verbatim replacement that allows latex
    \usepackage{grffile} % extends the file name processing of package graphics 
                         % to support a larger range 
    % The hyperref package gives us a pdf with properly built
    % internal navigation ('pdf bookmarks' for the table of contents,
    % internal cross-reference links, web links for URLs, etc.)
    \usepackage{hyperref}
    \usepackage{longtable} % longtable support required by pandoc >1.10
    \usepackage{booktabs}  % table support for pandoc > 1.12.2
    \usepackage[inline]{enumitem} % IRkernel/repr support (it uses the enumerate* environment)
    \usepackage[normalem]{ulem} % ulem is needed to support strikethroughs (\sout)
                                % normalem makes italics be italics, not underlines
    

    
    
    % Colors for the hyperref package
    \definecolor{urlcolor}{rgb}{0,.145,.698}
    \definecolor{linkcolor}{rgb}{.71,0.21,0.01}
    \definecolor{citecolor}{rgb}{.12,.54,.11}

    % ANSI colors
    \definecolor{ansi-black}{HTML}{3E424D}
    \definecolor{ansi-black-intense}{HTML}{282C36}
    \definecolor{ansi-red}{HTML}{E75C58}
    \definecolor{ansi-red-intense}{HTML}{B22B31}
    \definecolor{ansi-green}{HTML}{00A250}
    \definecolor{ansi-green-intense}{HTML}{007427}
    \definecolor{ansi-yellow}{HTML}{DDB62B}
    \definecolor{ansi-yellow-intense}{HTML}{B27D12}
    \definecolor{ansi-blue}{HTML}{208FFB}
    \definecolor{ansi-blue-intense}{HTML}{0065CA}
    \definecolor{ansi-magenta}{HTML}{D160C4}
    \definecolor{ansi-magenta-intense}{HTML}{A03196}
    \definecolor{ansi-cyan}{HTML}{60C6C8}
    \definecolor{ansi-cyan-intense}{HTML}{258F8F}
    \definecolor{ansi-white}{HTML}{C5C1B4}
    \definecolor{ansi-white-intense}{HTML}{A1A6B2}

    % commands and environments needed by pandoc snippets
    % extracted from the output of `pandoc -s`
    \providecommand{\tightlist}{%
      \setlength{\itemsep}{0pt}\setlength{\parskip}{0pt}}
    \DefineVerbatimEnvironment{Highlighting}{Verbatim}{commandchars=\\\{\}}
    % Add ',fontsize=\small' for more characters per line
    \newenvironment{Shaded}{}{}
    \newcommand{\KeywordTok}[1]{\textcolor[rgb]{0.00,0.44,0.13}{\textbf{{#1}}}}
    \newcommand{\DataTypeTok}[1]{\textcolor[rgb]{0.56,0.13,0.00}{{#1}}}
    \newcommand{\DecValTok}[1]{\textcolor[rgb]{0.25,0.63,0.44}{{#1}}}
    \newcommand{\BaseNTok}[1]{\textcolor[rgb]{0.25,0.63,0.44}{{#1}}}
    \newcommand{\FloatTok}[1]{\textcolor[rgb]{0.25,0.63,0.44}{{#1}}}
    \newcommand{\CharTok}[1]{\textcolor[rgb]{0.25,0.44,0.63}{{#1}}}
    \newcommand{\StringTok}[1]{\textcolor[rgb]{0.25,0.44,0.63}{{#1}}}
    \newcommand{\CommentTok}[1]{\textcolor[rgb]{0.38,0.63,0.69}{\textit{{#1}}}}
    \newcommand{\OtherTok}[1]{\textcolor[rgb]{0.00,0.44,0.13}{{#1}}}
    \newcommand{\AlertTok}[1]{\textcolor[rgb]{1.00,0.00,0.00}{\textbf{{#1}}}}
    \newcommand{\FunctionTok}[1]{\textcolor[rgb]{0.02,0.16,0.49}{{#1}}}
    \newcommand{\RegionMarkerTok}[1]{{#1}}
    \newcommand{\ErrorTok}[1]{\textcolor[rgb]{1.00,0.00,0.00}{\textbf{{#1}}}}
    \newcommand{\NormalTok}[1]{{#1}}
    
    % Additional commands for more recent versions of Pandoc
    \newcommand{\ConstantTok}[1]{\textcolor[rgb]{0.53,0.00,0.00}{{#1}}}
    \newcommand{\SpecialCharTok}[1]{\textcolor[rgb]{0.25,0.44,0.63}{{#1}}}
    \newcommand{\VerbatimStringTok}[1]{\textcolor[rgb]{0.25,0.44,0.63}{{#1}}}
    \newcommand{\SpecialStringTok}[1]{\textcolor[rgb]{0.73,0.40,0.53}{{#1}}}
    \newcommand{\ImportTok}[1]{{#1}}
    \newcommand{\DocumentationTok}[1]{\textcolor[rgb]{0.73,0.13,0.13}{\textit{{#1}}}}
    \newcommand{\AnnotationTok}[1]{\textcolor[rgb]{0.38,0.63,0.69}{\textbf{\textit{{#1}}}}}
    \newcommand{\CommentVarTok}[1]{\textcolor[rgb]{0.38,0.63,0.69}{\textbf{\textit{{#1}}}}}
    \newcommand{\VariableTok}[1]{\textcolor[rgb]{0.10,0.09,0.49}{{#1}}}
    \newcommand{\ControlFlowTok}[1]{\textcolor[rgb]{0.00,0.44,0.13}{\textbf{{#1}}}}
    \newcommand{\OperatorTok}[1]{\textcolor[rgb]{0.40,0.40,0.40}{{#1}}}
    \newcommand{\BuiltInTok}[1]{{#1}}
    \newcommand{\ExtensionTok}[1]{{#1}}
    \newcommand{\PreprocessorTok}[1]{\textcolor[rgb]{0.74,0.48,0.00}{{#1}}}
    \newcommand{\AttributeTok}[1]{\textcolor[rgb]{0.49,0.56,0.16}{{#1}}}
    \newcommand{\InformationTok}[1]{\textcolor[rgb]{0.38,0.63,0.69}{\textbf{\textit{{#1}}}}}
    \newcommand{\WarningTok}[1]{\textcolor[rgb]{0.38,0.63,0.69}{\textbf{\textit{{#1}}}}}
    
    
    % Define a nice break command that doesn't care if a line doesn't already
    % exist.
    \def\br{\hspace*{\fill} \\* }
    % Math Jax compatability definitions
    \def\gt{>}
    \def\lt{<}
    % Document parameters
    \title{Computation of Direct Horizontal Ultraviolet Irradiance}\subtitle{Variation with Solar Zenith Angle}
    
    
    

    % Pygments definitions
    
\makeatletter
\def\PY@reset{\let\PY@it=\relax \let\PY@bf=\relax%
    \let\PY@ul=\relax \let\PY@tc=\relax%
    \let\PY@bc=\relax \let\PY@ff=\relax}
\def\PY@tok#1{\csname PY@tok@#1\endcsname}
\def\PY@toks#1+{\ifx\relax#1\empty\else%
    \PY@tok{#1}\expandafter\PY@toks\fi}
\def\PY@do#1{\PY@bc{\PY@tc{\PY@ul{%
    \PY@it{\PY@bf{\PY@ff{#1}}}}}}}
\def\PY#1#2{\PY@reset\PY@toks#1+\relax+\PY@do{#2}}

\expandafter\def\csname PY@tok@gd\endcsname{\def\PY@tc##1{\textcolor[rgb]{0.63,0.00,0.00}{##1}}}
\expandafter\def\csname PY@tok@gu\endcsname{\let\PY@bf=\textbf\def\PY@tc##1{\textcolor[rgb]{0.50,0.00,0.50}{##1}}}
\expandafter\def\csname PY@tok@gt\endcsname{\def\PY@tc##1{\textcolor[rgb]{0.00,0.27,0.87}{##1}}}
\expandafter\def\csname PY@tok@gs\endcsname{\let\PY@bf=\textbf}
\expandafter\def\csname PY@tok@gr\endcsname{\def\PY@tc##1{\textcolor[rgb]{1.00,0.00,0.00}{##1}}}
\expandafter\def\csname PY@tok@cm\endcsname{\let\PY@it=\textit\def\PY@tc##1{\textcolor[rgb]{0.25,0.50,0.50}{##1}}}
\expandafter\def\csname PY@tok@vg\endcsname{\def\PY@tc##1{\textcolor[rgb]{0.10,0.09,0.49}{##1}}}
\expandafter\def\csname PY@tok@vi\endcsname{\def\PY@tc##1{\textcolor[rgb]{0.10,0.09,0.49}{##1}}}
\expandafter\def\csname PY@tok@vm\endcsname{\def\PY@tc##1{\textcolor[rgb]{0.10,0.09,0.49}{##1}}}
\expandafter\def\csname PY@tok@mh\endcsname{\def\PY@tc##1{\textcolor[rgb]{0.40,0.40,0.40}{##1}}}
\expandafter\def\csname PY@tok@cs\endcsname{\let\PY@it=\textit\def\PY@tc##1{\textcolor[rgb]{0.25,0.50,0.50}{##1}}}
\expandafter\def\csname PY@tok@ge\endcsname{\let\PY@it=\textit}
\expandafter\def\csname PY@tok@vc\endcsname{\def\PY@tc##1{\textcolor[rgb]{0.10,0.09,0.49}{##1}}}
\expandafter\def\csname PY@tok@il\endcsname{\def\PY@tc##1{\textcolor[rgb]{0.40,0.40,0.40}{##1}}}
\expandafter\def\csname PY@tok@go\endcsname{\def\PY@tc##1{\textcolor[rgb]{0.53,0.53,0.53}{##1}}}
\expandafter\def\csname PY@tok@cp\endcsname{\def\PY@tc##1{\textcolor[rgb]{0.74,0.48,0.00}{##1}}}
\expandafter\def\csname PY@tok@gi\endcsname{\def\PY@tc##1{\textcolor[rgb]{0.00,0.63,0.00}{##1}}}
\expandafter\def\csname PY@tok@gh\endcsname{\let\PY@bf=\textbf\def\PY@tc##1{\textcolor[rgb]{0.00,0.00,0.50}{##1}}}
\expandafter\def\csname PY@tok@ni\endcsname{\let\PY@bf=\textbf\def\PY@tc##1{\textcolor[rgb]{0.60,0.60,0.60}{##1}}}
\expandafter\def\csname PY@tok@nl\endcsname{\def\PY@tc##1{\textcolor[rgb]{0.63,0.63,0.00}{##1}}}
\expandafter\def\csname PY@tok@nn\endcsname{\let\PY@bf=\textbf\def\PY@tc##1{\textcolor[rgb]{0.00,0.00,1.00}{##1}}}
\expandafter\def\csname PY@tok@no\endcsname{\def\PY@tc##1{\textcolor[rgb]{0.53,0.00,0.00}{##1}}}
\expandafter\def\csname PY@tok@na\endcsname{\def\PY@tc##1{\textcolor[rgb]{0.49,0.56,0.16}{##1}}}
\expandafter\def\csname PY@tok@nb\endcsname{\def\PY@tc##1{\textcolor[rgb]{0.00,0.50,0.00}{##1}}}
\expandafter\def\csname PY@tok@nc\endcsname{\let\PY@bf=\textbf\def\PY@tc##1{\textcolor[rgb]{0.00,0.00,1.00}{##1}}}
\expandafter\def\csname PY@tok@nd\endcsname{\def\PY@tc##1{\textcolor[rgb]{0.67,0.13,1.00}{##1}}}
\expandafter\def\csname PY@tok@ne\endcsname{\let\PY@bf=\textbf\def\PY@tc##1{\textcolor[rgb]{0.82,0.25,0.23}{##1}}}
\expandafter\def\csname PY@tok@nf\endcsname{\def\PY@tc##1{\textcolor[rgb]{0.00,0.00,1.00}{##1}}}
\expandafter\def\csname PY@tok@si\endcsname{\let\PY@bf=\textbf\def\PY@tc##1{\textcolor[rgb]{0.73,0.40,0.53}{##1}}}
\expandafter\def\csname PY@tok@s2\endcsname{\def\PY@tc##1{\textcolor[rgb]{0.73,0.13,0.13}{##1}}}
\expandafter\def\csname PY@tok@nt\endcsname{\let\PY@bf=\textbf\def\PY@tc##1{\textcolor[rgb]{0.00,0.50,0.00}{##1}}}
\expandafter\def\csname PY@tok@nv\endcsname{\def\PY@tc##1{\textcolor[rgb]{0.10,0.09,0.49}{##1}}}
\expandafter\def\csname PY@tok@s1\endcsname{\def\PY@tc##1{\textcolor[rgb]{0.73,0.13,0.13}{##1}}}
\expandafter\def\csname PY@tok@dl\endcsname{\def\PY@tc##1{\textcolor[rgb]{0.73,0.13,0.13}{##1}}}
\expandafter\def\csname PY@tok@ch\endcsname{\let\PY@it=\textit\def\PY@tc##1{\textcolor[rgb]{0.25,0.50,0.50}{##1}}}
\expandafter\def\csname PY@tok@m\endcsname{\def\PY@tc##1{\textcolor[rgb]{0.40,0.40,0.40}{##1}}}
\expandafter\def\csname PY@tok@gp\endcsname{\let\PY@bf=\textbf\def\PY@tc##1{\textcolor[rgb]{0.00,0.00,0.50}{##1}}}
\expandafter\def\csname PY@tok@sh\endcsname{\def\PY@tc##1{\textcolor[rgb]{0.73,0.13,0.13}{##1}}}
\expandafter\def\csname PY@tok@ow\endcsname{\let\PY@bf=\textbf\def\PY@tc##1{\textcolor[rgb]{0.67,0.13,1.00}{##1}}}
\expandafter\def\csname PY@tok@sx\endcsname{\def\PY@tc##1{\textcolor[rgb]{0.00,0.50,0.00}{##1}}}
\expandafter\def\csname PY@tok@bp\endcsname{\def\PY@tc##1{\textcolor[rgb]{0.00,0.50,0.00}{##1}}}
\expandafter\def\csname PY@tok@c1\endcsname{\let\PY@it=\textit\def\PY@tc##1{\textcolor[rgb]{0.25,0.50,0.50}{##1}}}
\expandafter\def\csname PY@tok@fm\endcsname{\def\PY@tc##1{\textcolor[rgb]{0.00,0.00,1.00}{##1}}}
\expandafter\def\csname PY@tok@o\endcsname{\def\PY@tc##1{\textcolor[rgb]{0.40,0.40,0.40}{##1}}}
\expandafter\def\csname PY@tok@kc\endcsname{\let\PY@bf=\textbf\def\PY@tc##1{\textcolor[rgb]{0.00,0.50,0.00}{##1}}}
\expandafter\def\csname PY@tok@c\endcsname{\let\PY@it=\textit\def\PY@tc##1{\textcolor[rgb]{0.25,0.50,0.50}{##1}}}
\expandafter\def\csname PY@tok@mf\endcsname{\def\PY@tc##1{\textcolor[rgb]{0.40,0.40,0.40}{##1}}}
\expandafter\def\csname PY@tok@err\endcsname{\def\PY@bc##1{\setlength{\fboxsep}{0pt}\fcolorbox[rgb]{1.00,0.00,0.00}{1,1,1}{\strut ##1}}}
\expandafter\def\csname PY@tok@mb\endcsname{\def\PY@tc##1{\textcolor[rgb]{0.40,0.40,0.40}{##1}}}
\expandafter\def\csname PY@tok@ss\endcsname{\def\PY@tc##1{\textcolor[rgb]{0.10,0.09,0.49}{##1}}}
\expandafter\def\csname PY@tok@sr\endcsname{\def\PY@tc##1{\textcolor[rgb]{0.73,0.40,0.53}{##1}}}
\expandafter\def\csname PY@tok@mo\endcsname{\def\PY@tc##1{\textcolor[rgb]{0.40,0.40,0.40}{##1}}}
\expandafter\def\csname PY@tok@kd\endcsname{\let\PY@bf=\textbf\def\PY@tc##1{\textcolor[rgb]{0.00,0.50,0.00}{##1}}}
\expandafter\def\csname PY@tok@mi\endcsname{\def\PY@tc##1{\textcolor[rgb]{0.40,0.40,0.40}{##1}}}
\expandafter\def\csname PY@tok@kn\endcsname{\let\PY@bf=\textbf\def\PY@tc##1{\textcolor[rgb]{0.00,0.50,0.00}{##1}}}
\expandafter\def\csname PY@tok@cpf\endcsname{\let\PY@it=\textit\def\PY@tc##1{\textcolor[rgb]{0.25,0.50,0.50}{##1}}}
\expandafter\def\csname PY@tok@kr\endcsname{\let\PY@bf=\textbf\def\PY@tc##1{\textcolor[rgb]{0.00,0.50,0.00}{##1}}}
\expandafter\def\csname PY@tok@s\endcsname{\def\PY@tc##1{\textcolor[rgb]{0.73,0.13,0.13}{##1}}}
\expandafter\def\csname PY@tok@kp\endcsname{\def\PY@tc##1{\textcolor[rgb]{0.00,0.50,0.00}{##1}}}
\expandafter\def\csname PY@tok@w\endcsname{\def\PY@tc##1{\textcolor[rgb]{0.73,0.73,0.73}{##1}}}
\expandafter\def\csname PY@tok@kt\endcsname{\def\PY@tc##1{\textcolor[rgb]{0.69,0.00,0.25}{##1}}}
\expandafter\def\csname PY@tok@sc\endcsname{\def\PY@tc##1{\textcolor[rgb]{0.73,0.13,0.13}{##1}}}
\expandafter\def\csname PY@tok@sb\endcsname{\def\PY@tc##1{\textcolor[rgb]{0.73,0.13,0.13}{##1}}}
\expandafter\def\csname PY@tok@sa\endcsname{\def\PY@tc##1{\textcolor[rgb]{0.73,0.13,0.13}{##1}}}
\expandafter\def\csname PY@tok@k\endcsname{\let\PY@bf=\textbf\def\PY@tc##1{\textcolor[rgb]{0.00,0.50,0.00}{##1}}}
\expandafter\def\csname PY@tok@se\endcsname{\let\PY@bf=\textbf\def\PY@tc##1{\textcolor[rgb]{0.73,0.40,0.13}{##1}}}
\expandafter\def\csname PY@tok@sd\endcsname{\let\PY@it=\textit\def\PY@tc##1{\textcolor[rgb]{0.73,0.13,0.13}{##1}}}

\def\PYZbs{\char`\\}
\def\PYZus{\char`\_}
\def\PYZob{\char`\{}
\def\PYZcb{\char`\}}
\def\PYZca{\char`\^}
\def\PYZam{\char`\&}
\def\PYZlt{\char`\<}
\def\PYZgt{\char`\>}
\def\PYZsh{\char`\#}
\def\PYZpc{\char`\%}
\def\PYZdl{\char`\$}
\def\PYZhy{\char`\-}
\def\PYZsq{\char`\'}
\def\PYZdq{\char`\"}
\def\PYZti{\char`\~}
% for compatibility with earlier versions
\def\PYZat{@}
\def\PYZlb{[}
\def\PYZrb{]}
\makeatother


    % Exact colors from NB
    \definecolor{incolor}{rgb}{0.0, 0.0, 0.5}
    \definecolor{outcolor}{rgb}{0.545, 0.0, 0.0}



    
    % Prevent overflowing lines due to hard-to-break entities
    \sloppy 
    % Setup hyperref package
    \hypersetup{
      breaklinks=true,  % so long urls are correctly broken across lines
      colorlinks=true,
      urlcolor=urlcolor,
      linkcolor=linkcolor,
      citecolor=citecolor,
      }
    % Slightly bigger margins than the latex defaults
    
    \geometry{verbose,tmargin=1in,bmargin=1in,lmargin=1in,rmargin=1in}
    
    

    \begin{document}
    
    
    \maketitle
    
    

    
    \section{Using libRadtran in Parallel on a
Cluster}\label{using-libradtran-in-parallel-on-a-cluster}

    This notebook shows how to run a batch of
\href{http://www.libradtran.org}{\texttt{libRadtran/uvspec}} cases in
parallel on a cluster computer using
\href{http://distributed.readthedocs.io/en/latest/limitations.html}{\texttt{dask.distributed}}.
The \texttt{dask.distributed} package provides a flexible, yet
reasonably simple way of harnessing mutiple compute cores from the
\texttt{IPython/Jupyter} notebook environment or in other Python launch
modes.

It is somewhat easier to set up and use the \texttt{dask.distributed}
cluster client than the \texttt{ipyparallel} client. However,
\texttt{dask.distributed} provides even less security and must only be
run on local trusted networks.

This particular example computes the direct horizontal ultraviolet B
(UVB, 300 nm to 340 nm) irradiance at Bottom-Of-Atmosphere (BOA) as a
function of the solar zenith angle (SZA).

    \begin{Verbatim}[commandchars=\\\{\}]
{\color{incolor}In [{\color{incolor}40}]:} \PY{k+kn}{import} \PY{n+nn}{morticia.rad.librad} \PY{k+kn}{as} \PY{n+nn}{librad}
         \PY{c+c1}{\PYZsh{} Use auto reload of librad for development purposes}
         \PY{o}{\PYZpc{}}\PY{k}{load\PYZus{}ext} autoreload
         \PY{o}{\PYZpc{}}\PY{k}{autoreload} 1
         \PY{o}{\PYZpc{}}\PY{k}{aimport} morticia.rad.librad
         \PY{k+kn}{import} \PY{n+nn}{numpy} \PY{k+kn}{as} \PY{n+nn}{np}
         \PY{k+kn}{import} \PY{n+nn}{matplotlib} \PY{k+kn}{as} \PY{n+nn}{mpl}
         \PY{k+kn}{import} \PY{n+nn}{matplotlib.pyplot} \PY{k+kn}{as} \PY{n+nn}{plt}
         \PY{k+kn}{from} \PY{n+nn}{dask.distributed} \PY{k+kn}{import} \PY{n}{Client} \PY{c+c1}{\PYZsh{} For contacting the dask scheduler}
         \PY{c+c1}{\PYZsh{} use latex for font rendering}
         \PY{n}{mpl}\PY{o}{.}\PY{n}{rcParams}\PY{p}{[}\PY{l+s+s1}{\PYZsq{}}\PY{l+s+s1}{text.usetex}\PY{l+s+s1}{\PYZsq{}}\PY{p}{]} \PY{o}{=} \PY{n+nb+bp}{True}  \PY{c+c1}{\PYZsh{} Use TeX to format labels (takes a bit longer)}
         \PY{o}{\PYZpc{}}\PY{k}{matplotlib} inline
\end{Verbatim}


    \begin{Verbatim}[commandchars=\\\{\}]
The autoreload extension is already loaded. To reload it, use:
  \%reload\_ext autoreload

    \end{Verbatim}

    \begin{Verbatim}[commandchars=\\\{\}]
{\color{incolor}In [{\color{incolor}41}]:} \PY{c+c1}{\PYZsh{} Load a libRadtran example case}
         \PY{c+c1}{\PYZsh{} Be default, any include files are expanded, creating a single set of option keywords}
         \PY{n}{libRadCase}\PY{o}{=}\PY{n}{librad}\PY{o}{.}\PY{n}{Case}\PY{p}{(}\PY{n}{filename}\PY{o}{=}\PY{l+s+s1}{\PYZsq{}}\PY{l+s+s1}{./examples/UVSPEC\PYZus{}AEROSOL.INP}\PY{l+s+s1}{\PYZsq{}}\PY{p}{)}
         \PY{n}{libRadCase}\PY{o}{.}\PY{n}{purge} \PY{o}{=} \PY{n+nb+bp}{True}
\end{Verbatim}


    \begin{Verbatim}[commandchars=\\\{\}]
{\color{incolor}In [{\color{incolor}42}]:} \PY{c+c1}{\PYZsh{} Show the input of the basecase}
         \PY{n}{libRadCase}
\end{Verbatim}


\begin{Verbatim}[commandchars=\\\{\}]
{\color{outcolor}Out[{\color{outcolor}42}]:} atmosphere\_file ../data/atmmod/afglus.dat
         source solar ../data/solar\_flux/atlas\_plus\_modtran
         mol\_modify O3 300. DU
         day\_of\_year 170
         albedo 0.2
         sza 32.0
         rte\_solver disort
         number\_of\_streams 6
         wavelength 299.0 341.0
         slit\_function\_file ../examples/TRI\_SLIT.DAT
         spline 300 340 1
         quiet 
         aerosol\_vulcan 1
         aerosol\_haze 6
         aerosol\_season 1
         aerosol\_visibility 20.0
         aerosol\_angstrom 1.1 0.2
         aerosol\_modify ssa scale 0.85
         aerosol\_modify gg set 0.70
         aerosol\_file tau ../examples/AERO\_TAU.DAT
\end{Verbatim}
            
    \begin{Verbatim}[commandchars=\\\{\}]
{\color{incolor}In [{\color{incolor}43}]:} \PY{c+c1}{\PYZsh{} Create 10 copies of the case in a list, with different solar zenith angles}
         \PY{k+kn}{import} \PY{n+nn}{copy}
         \PY{n}{libRadBatch} \PY{o}{=} \PY{p}{[}\PY{n}{copy}\PY{o}{.}\PY{n}{deepcopy}\PY{p}{(}\PY{n}{libRadCase}\PY{p}{)} \PY{k}{for} \PY{n}{icopy} \PY{o+ow}{in} \PY{n+nb}{range}\PY{p}{(}\PY{l+m+mi}{9}\PY{p}{)}\PY{p}{]}
         \PY{c+c1}{\PYZsh{} Rename each element in the batch, just by appending a digit}
         \PY{c+c1}{\PYZsh{} This is to prevent the engines from fighting over input and output files}
         \PY{c+c1}{\PYZsh{} Also change the solar zenith angle on each run, so that they are different}
         \PY{n}{solar\PYZus{}zenith\PYZus{}angle} \PY{o}{=} \PY{n}{np}\PY{o}{.}\PY{n}{linspace}\PY{p}{(}\PY{l+m+mf}{0.0}\PY{p}{,} \PY{l+m+mf}{80.0}\PY{p}{,} \PY{l+m+mi}{9}\PY{p}{)}
         \PY{k}{for} \PY{n}{icase}\PY{p}{,} \PY{n}{libRadCase} \PY{o+ow}{in} \PY{n+nb}{enumerate}\PY{p}{(}\PY{n}{libRadBatch}\PY{p}{)}\PY{p}{:}
             \PY{n}{libRadCase}\PY{o}{.}\PY{n}{name} \PY{o}{=} \PY{n}{libRadCase}\PY{o}{.}\PY{n}{name} \PY{o}{+} \PY{l+s+s1}{\PYZsq{}}\PY{l+s+s1}{\PYZus{}}\PY{l+s+s1}{\PYZsq{}} \PY{o}{+} \PY{n+nb}{str}\PY{p}{(}\PY{n}{icase}\PY{p}{)}
             \PY{n}{libRadCase}\PY{o}{.}\PY{n}{alter\PYZus{}option}\PY{p}{(}\PY{p}{[}\PY{l+s+s1}{\PYZsq{}}\PY{l+s+s1}{sza}\PY{l+s+s1}{\PYZsq{}}\PY{p}{,} \PY{n+nb}{str}\PY{p}{(}\PY{n}{solar\PYZus{}zenith\PYZus{}angle}\PY{p}{[}\PY{n}{icase}\PY{p}{]}\PY{p}{)}\PY{p}{]}\PY{p}{)}
\end{Verbatim}


    \begin{Verbatim}[commandchars=\\\{\}]
{\color{incolor}In [{\color{incolor}44}]:} \PY{c+c1}{\PYZsh{} Show the first and second cases in the list}
         \PY{k}{print}\PY{p}{(}\PY{n}{libRadBatch}\PY{p}{[}\PY{l+m+mi}{0}\PY{p}{]}\PY{p}{)}
         \PY{k}{print}
         \PY{k}{print}\PY{p}{(}\PY{n}{libRadBatch}\PY{p}{[}\PY{l+m+mi}{1}\PY{p}{]}\PY{p}{)}
\end{Verbatim}


    \begin{Verbatim}[commandchars=\\\{\}]
atmosphere\_file ../data/atmmod/afglus.dat
source solar ../data/solar\_flux/atlas\_plus\_modtran
mol\_modify O3 300. DU
day\_of\_year 170
albedo 0.2
sza 0.0
rte\_solver disort
number\_of\_streams 6
wavelength 299.0 341.0
slit\_function\_file ../examples/TRI\_SLIT.DAT
spline 300 340 1
quiet 
aerosol\_vulcan 1
aerosol\_haze 6
aerosol\_season 1
aerosol\_visibility 20.0
aerosol\_angstrom 1.1 0.2
aerosol\_modify ssa scale 0.85
aerosol\_modify gg set 0.70
aerosol\_file tau ../examples/AERO\_TAU.DAT

atmosphere\_file ../data/atmmod/afglus.dat
source solar ../data/solar\_flux/atlas\_plus\_modtran
mol\_modify O3 300. DU
day\_of\_year 170
albedo 0.2
sza 10.0
rte\_solver disort
number\_of\_streams 6
wavelength 299.0 341.0
slit\_function\_file ../examples/TRI\_SLIT.DAT
spline 300 340 1
quiet 
aerosol\_vulcan 1
aerosol\_haze 6
aerosol\_season 1
aerosol\_visibility 20.0
aerosol\_angstrom 1.1 0.2
aerosol\_modify ssa scale 0.85
aerosol\_modify gg set 0.70
aerosol\_file tau ../examples/AERO\_TAU.DAT

    \end{Verbatim}

    \begin{Verbatim}[commandchars=\\\{\}]
{\color{incolor}In [{\color{incolor}45}]:} \PY{c+c1}{\PYZsh{} Create a compute client calling the dask Client method with IP }
         \PY{c+c1}{\PYZsh{} and port of scheduler}
         \PY{n}{paraclient} \PY{o}{=} \PY{n}{Client}\PY{p}{(}\PY{l+s+s1}{\PYZsq{}}\PY{l+s+s1}{146.64.246.94:8786}\PY{l+s+s1}{\PYZsq{}}\PY{p}{)}
\end{Verbatim}


    \begin{Verbatim}[commandchars=\\\{\}]
{\color{incolor}In [{\color{incolor}46}]:} \PY{c+c1}{\PYZsh{} Now try to run the batch on the cluster}
         \PY{n}{futureRadBatch} \PY{o}{=} \PY{n}{paraclient}\PY{o}{.}\PY{n}{map}\PY{p}{(}\PY{n}{librad}\PY{o}{.}\PY{n}{Case}\PY{o}{.}\PY{n}{run}\PY{p}{,} \PY{n}{libRadBatch}\PY{p}{)}
\end{Verbatim}


    \begin{Verbatim}[commandchars=\\\{\}]
{\color{incolor}In [{\color{incolor}47}]:} \PY{c+c1}{\PYZsh{} Gather results. This will wait for completion of all tasks.}
         \PY{n}{libRadBatch} \PY{o}{=} \PY{n}{paraclient}\PY{o}{.}\PY{n}{gather}\PY{p}{(}\PY{n}{futureRadBatch}\PY{p}{)}
\end{Verbatim}


    \begin{Verbatim}[commandchars=\\\{\}]
{\color{incolor}In [{\color{incolor}48}]:} \PY{c+c1}{\PYZsh{} Obtain the direct solar spectral irradiance}
         \PY{c+c1}{\PYZsh{} Compile all results into a numpy array}
         \PY{n}{edir} \PY{o}{=} \PY{n}{np}\PY{o}{.}\PY{n}{hstack}\PY{p}{(}\PY{p}{[}\PY{n}{libRadBatch}\PY{p}{[}\PY{n}{i}\PY{p}{]}\PY{o}{.}\PY{n}{edir} \PY{k}{for} \PY{n}{i} \PY{o+ow}{in} \PY{n+nb}{range}\PY{p}{(}\PY{n+nb}{len}\PY{p}{(}\PY{n}{libRadBatch}\PY{p}{)}\PY{p}{)}\PY{p}{]}\PY{p}{)} 
         \PY{n}{edir}\PY{o}{.}\PY{n}{shape}
\end{Verbatim}


\begin{Verbatim}[commandchars=\\\{\}]
{\color{outcolor}Out[{\color{outcolor}48}]:} (41L, 9L)
\end{Verbatim}
            
    \begin{Verbatim}[commandchars=\\\{\}]
{\color{incolor}In [{\color{incolor}49}]:} \PY{c+c1}{\PYZsh{} Plot direct irradiance for all 10 cases}
         \PY{n}{plt}\PY{o}{.}\PY{n}{figure}\PY{p}{(}\PY{n}{figsize}\PY{o}{=}\PY{p}{(}\PY{l+m+mi}{8}\PY{p}{,}\PY{l+m+mi}{6}\PY{p}{)}\PY{p}{)}
         \PY{n}{plt}\PY{o}{.}\PY{n}{plot}\PY{p}{(}\PY{n}{libRadBatch}\PY{p}{[}\PY{l+m+mi}{0}\PY{p}{]}\PY{o}{.}\PY{n}{wvl}\PY{p}{,} \PY{n}{edir}\PY{p}{)}
         \PY{n}{plt}\PY{o}{.}\PY{n}{title}\PY{p}{(}\PY{l+s+s1}{\PYZsq{}}\PY{l+s+s1}{Direct Horizontal Irradiance as a function of Solar Zenith Angle}\PY{l+s+s1}{\PYZsq{}}\PY{p}{)}
         \PY{n}{plt}\PY{o}{.}\PY{n}{xlabel}\PY{p}{(}\PY{l+s+s1}{\PYZsq{}}\PY{l+s+s1}{Wavelenth [nm]}\PY{l+s+s1}{\PYZsq{}}\PY{p}{)}
         \PY{n}{plt}\PY{o}{.}\PY{n}{ylabel}\PY{p}{(}\PY{l+s+s1}{\PYZsq{}}\PY{l+s+s1}{Direct Horizontal Irradiance [\PYZdl{}}\PY{l+s+s1}{\PYZsq{}} \PY{o}{+} 
                    \PY{l+s+s1}{\PYZsq{}}\PY{l+s+s1}{/}\PY{l+s+s1}{\PYZsq{}}\PY{o}{.}\PY{n}{join}\PY{p}{(}\PY{n}{libRadBatch}\PY{p}{[}\PY{l+m+mi}{0}\PY{p}{]}\PY{o}{.}\PY{n}{rad\PYZus{}units}\PY{p}{)}\PY{o}{+} \PY{l+s+s1}{\PYZsq{}}\PY{l+s+s1}{\PYZdl{}]}\PY{l+s+s1}{\PYZsq{}}\PY{p}{)}
         \PY{n}{the\PYZus{}legends} \PY{o}{=} \PY{p}{[}\PY{l+s+s1}{\PYZsq{}}\PY{l+s+s1}{Solar Zenith Angle : }\PY{l+s+s1}{\PYZsq{}} \PY{o}{+} 
                        \PY{n+nb}{str}\PY{p}{(}\PY{n}{leg}\PY{p}{)} \PY{o}{+} \PY{l+s+s1}{\PYZsq{}}\PY{l+s+s1}{\PYZdl{}\PYZca{}}\PY{l+s+s1}{\PYZbs{}}\PY{l+s+s1}{circ\PYZdl{}}\PY{l+s+s1}{\PYZsq{}} \PY{k}{for} \PY{n}{leg} \PY{o+ow}{in} 
                        \PY{n}{solar\PYZus{}zenith\PYZus{}angle}\PY{p}{]}
         \PY{n}{plt}\PY{o}{.}\PY{n}{legend}\PY{p}{(}\PY{n}{the\PYZus{}legends}\PY{p}{)}
         \PY{n}{plt}\PY{o}{.}\PY{n}{grid}\PY{p}{(}\PY{p}{)}
\end{Verbatim}


    \begin{center}
    \adjustimage{max size={0.9\linewidth}{0.9\paperheight}}{05b-libRadtran-dask-distributed_files/05b-libRadtran-dask-distributed_11_0.png}
    \end{center}
    { \hspace*{\fill} \\}
    
    \begin{Verbatim}[commandchars=\\\{\}]
{\color{incolor}In [{\color{incolor}54}]:} \PY{c+c1}{\PYZsh{} Repeat plot, but on log scale}
         \PY{c+c1}{\PYZsh{} Plot direct irradiance for all 10 cases}
         \PY{n}{plt}\PY{o}{.}\PY{n}{figure}\PY{p}{(}\PY{n}{figsize}\PY{o}{=}\PY{p}{(}\PY{l+m+mi}{8}\PY{p}{,}\PY{l+m+mi}{6}\PY{p}{)}\PY{p}{)}
         \PY{n}{plt}\PY{o}{.}\PY{n}{semilogy}\PY{p}{(}\PY{n}{libRadBatch}\PY{p}{[}\PY{l+m+mi}{0}\PY{p}{]}\PY{o}{.}\PY{n}{wvl}\PY{p}{,} \PY{n}{edir}\PY{p}{)}
         \PY{n}{plt}\PY{o}{.}\PY{n}{title}\PY{p}{(}\PY{l+s+s1}{\PYZsq{}}\PY{l+s+s1}{Direct Horizontal Irradiance as a function of Solar Zenith Angle}\PY{l+s+s1}{\PYZsq{}}\PY{p}{)}
         \PY{n}{plt}\PY{o}{.}\PY{n}{xlabel}\PY{p}{(}\PY{l+s+s1}{\PYZsq{}}\PY{l+s+s1}{Wavelenth [nm]}\PY{l+s+s1}{\PYZsq{}}\PY{p}{)}
         \PY{n}{plt}\PY{o}{.}\PY{n}{ylabel}\PY{p}{(}\PY{l+s+s1}{\PYZsq{}}\PY{l+s+s1}{Direct Horizontal Irradiance [\PYZdl{}}\PY{l+s+s1}{\PYZsq{}} \PY{o}{+} 
                    \PY{l+s+s1}{\PYZsq{}}\PY{l+s+s1}{/}\PY{l+s+s1}{\PYZsq{}}\PY{o}{.}\PY{n}{join}\PY{p}{(}\PY{n}{libRadBatch}\PY{p}{[}\PY{l+m+mi}{0}\PY{p}{]}\PY{o}{.}\PY{n}{rad\PYZus{}units}\PY{p}{)}\PY{o}{+} \PY{l+s+s1}{\PYZsq{}}\PY{l+s+s1}{\PYZdl{}]}\PY{l+s+s1}{\PYZsq{}}\PY{p}{)}
         \PY{n}{the\PYZus{}legends} \PY{o}{=} \PY{p}{[}\PY{l+s+s1}{\PYZsq{}}\PY{l+s+s1}{Solar Zenith Angle : }\PY{l+s+s1}{\PYZsq{}} \PY{o}{+} \PY{n+nb}{str}\PY{p}{(}\PY{n}{leg}\PY{p}{)} \PY{o}{+} 
                        \PY{l+s+s1}{\PYZsq{}}\PY{l+s+s1}{\PYZdl{}\PYZca{}}\PY{l+s+s1}{\PYZbs{}}\PY{l+s+s1}{circ\PYZdl{}}\PY{l+s+s1}{\PYZsq{}} \PY{k}{for} \PY{n}{leg} \PY{o+ow}{in} \PY{n}{solar\PYZus{}zenith\PYZus{}angle}\PY{p}{]}
         \PY{n}{plt}\PY{o}{.}\PY{n}{legend}\PY{p}{(}\PY{n}{the\PYZus{}legends}\PY{p}{)}
         \PY{n}{plt}\PY{o}{.}\PY{n}{grid}\PY{p}{(}\PY{p}{)}
\end{Verbatim}


    \begin{center}
    \adjustimage{max size={0.9\linewidth}{0.9\paperheight}}{05b-libRadtran-dask-distributed_files/05b-libRadtran-dask-distributed_12_0.png}
    \end{center}
    { \hspace*{\fill} \\}
    
    \begin{Verbatim}[commandchars=\\\{\}]
{\color{incolor}In [{\color{incolor}55}]:} \PY{k+kn}{import} \PY{n+nn}{datetime}
         \PY{n}{now} \PY{o}{=} \PY{n}{datetime}\PY{o}{.}\PY{n}{datetime}\PY{o}{.}\PY{n}{now}\PY{p}{(}\PY{p}{)}
         \PY{k}{print} \PY{l+s+s1}{\PYZsq{}}\PY{l+s+s1}{Completed Run at }\PY{l+s+s1}{\PYZsq{}}\PY{p}{,} \PY{n+nb}{str}\PY{p}{(}\PY{n}{now}\PY{p}{)}
         \PY{c+c1}{\PYZsh{} After executing this cell, save the notebook before running the}
         \PY{c+c1}{\PYZsh{} publish cells below}
         \PY{c+c1}{\PYZsh{} To run the complete notebook, select this cell}
         \PY{c+c1}{\PYZsh{} and then choose Cell\PYZhy{}\PYZgt{}Run All Above from the menu}
\end{Verbatim}


    \begin{Verbatim}[commandchars=\\\{\}]
Completed Run at  2018-03-15 12:06:51.111000

    \end{Verbatim}

    \begin{Verbatim}[commandchars=\\\{\}]
{\color{incolor}In [{\color{incolor}52}]:} \PY{o}{\PYZpc{}}\PY{o}{\PYZpc{}}\PY{n+nx}{javascript}
         \PY{k+kd}{var} \PY{n+nx}{kernel} \PY{o}{=} \PY{n+nx}{IPython}\PY{p}{.}\PY{n+nx}{notebook}\PY{p}{.}\PY{n+nx}{kernel}\PY{p}{;}
         \PY{k+kd}{var} \PY{n+nx}{thename} \PY{o}{=} \PY{n+nb}{window}\PY{p}{.}\PY{n+nb}{document}\PY{p}{.}\PY{n+nx}{getElementById}\PY{p}{(}\PY{l+s+s2}{\PYZdq{}notebook\PYZus{}name\PYZdq{}}\PY{p}{)}\PY{p}{.}\PY{n+nx}{innerHTML}\PY{p}{;}
         \PY{k+kd}{var} \PY{n+nx}{command} \PY{o}{=} \PY{l+s+s2}{\PYZdq{}theNotebook = \PYZdq{}} \PY{o}{+} \PY{l+s+s2}{\PYZdq{}\PYZsq{}\PYZdq{}} \PY{o}{+} \PY{n+nx}{thename} \PY{o}{+} \PY{l+s+s2}{\PYZdq{}\PYZsq{}\PYZdq{}}\PY{p}{;}
         \PY{n+nx}{kernel}\PY{p}{.}\PY{n+nx}{execute}\PY{p}{(}\PY{n+nx}{command}\PY{p}{)}\PY{p}{;}
\end{Verbatim}


    
    \begin{verbatim}
<IPython.core.display.Javascript object>
    \end{verbatim}

    
    \begin{Verbatim}[commandchars=\\\{\}]
{\color{incolor}In [{\color{incolor}53}]:} \PY{n}{publish} \PY{o}{=} \PY{n+nb+bp}{True}
         \PY{n}{casename} \PY{o}{=} \PY{l+s+s1}{\PYZsq{}}\PY{l+s+s1}{Dask.Distributed.Example}\PY{l+s+s1}{\PYZsq{}}
         \PY{n}{casetitle} \PY{o}{=} \PY{l+s+s1}{\PYZsq{}}\PY{l+s+s1}{Computation of Direct Horizontal Ultraviolet Irradiance}\PY{l+s+s1}{\PYZsq{}}
         \PY{n}{subtitle} \PY{o}{=} \PY{l+s+s1}{\PYZsq{}}\PY{l+s+s1}{Variation with Solar Zenith Angle}\PY{l+s+s1}{\PYZsq{}}
         \PY{n}{result\PYZus{}folder} \PY{o}{=} \PY{l+s+s1}{\PYZsq{}}\PY{l+s+s1}{d:/Projects/MORTICIA/Docs/ReportFY201718/Appendices}\PY{l+s+s1}{\PYZsq{}}
         \PY{c+c1}{\PYZsh{} Run this cell to publish, but save the notebook beforehand}
         \PY{c+c1}{\PYZsh{} The contents of this cell strictly for publishing the notebook.}
         \PY{k}{if} \PY{n}{publish}\PY{p}{:}
             \PY{n}{notebook\PYZus{}name} \PY{o}{=} \PY{n}{theNotebook} \PY{o}{+} \PY{l+s+s1}{\PYZsq{}}\PY{l+s+s1}{.ipynb}\PY{l+s+s1}{\PYZsq{}}
             \PY{c+c1}{\PYZsh{} Run nbconvert to create a tex file as well as pdf graphic files}
             \PY{o}{!}jupyter nbconvert \PYZhy{}\PYZhy{}to latex \PY{n+nv}{\PYZdl{}notebook\PYZus{}name}
             \PY{c+c1}{\PYZsh{} Do touchups}
             \PY{k}{def} \PY{n+nf}{touchup\PYZus{}build\PYZus{}tex}\PY{p}{(}\PY{n}{tex\PYZus{}file}\PY{p}{,} \PY{n}{touchups}\PY{p}{)}\PY{p}{:}
                 \PY{c+c1}{\PYZsh{} Read the LaTeX file}
                 \PY{n}{f} \PY{o}{=} \PY{n+nb}{open}\PY{p}{(}\PY{n}{tex\PYZus{}file}\PY{p}{,}\PY{l+s+s1}{\PYZsq{}}\PY{l+s+s1}{r}\PY{l+s+s1}{\PYZsq{}}\PY{p}{)}
                 \PY{n}{filedata} \PY{o}{=} \PY{n}{f}\PY{o}{.}\PY{n}{read}\PY{p}{(}\PY{p}{)}
                 \PY{n}{f}\PY{o}{.}\PY{n}{close}\PY{p}{(}\PY{p}{)}
                 \PY{c+c1}{\PYZsh{} Perform touchups}
                 \PY{k}{for} \PY{n}{src}\PY{p}{,} \PY{n}{target} \PY{o+ow}{in} \PY{n}{touchups}\PY{o}{.}\PY{n}{iteritems}\PY{p}{(}\PY{p}{)}\PY{p}{:}
                     \PY{n}{filedata} \PY{o}{=} \PY{n}{filedata}\PY{o}{.}\PY{n}{replace}\PY{p}{(}\PY{n}{src}\PY{p}{,} \PY{n}{target}\PY{p}{)}
                 \PY{c+c1}{\PYZsh{} Write file again}
                 \PY{n}{f} \PY{o}{=} \PY{n+nb}{open}\PY{p}{(}\PY{n}{tex\PYZus{}file}\PY{p}{,}\PY{l+s+s1}{\PYZsq{}}\PY{l+s+s1}{w}\PY{l+s+s1}{\PYZsq{}}\PY{p}{)}
                 \PY{n}{f}\PY{o}{.}\PY{n}{write}\PY{p}{(}\PY{n}{filedata}\PY{p}{)}
                 \PY{n}{f}\PY{o}{.}\PY{n}{close}\PY{p}{(}\PY{p}{)}
                 \PY{c+c1}{\PYZsh{} Build PDF using LaTeX}
                 \PY{o}{!}pdflatex \PY{n+nv}{\PYZdl{}tex\PYZus{}file} \PYZgt{} pdflatex.out
                 
             \PY{n}{touchups} \PY{o}{=} \PY{p}{\PYZob{}}\PY{l+s+s1}{\PYZsq{}}\PY{l+s+s1}{[11pt]\PYZob{}article\PYZcb{}}\PY{l+s+s1}{\PYZsq{}}\PY{p}{:} \PY{l+s+s1}{\PYZsq{}}\PY{l+s+s1}{[11pt, a4paper, landscape]\PYZob{}scrartcl\PYZcb{}}\PY{l+s+s1}{\PYZsq{}}\PY{p}{,}
                     \PY{l+s+s1}{\PYZsq{}}\PY{l+s+s1}{title\PYZob{}}\PY{l+s+s1}{\PYZsq{}} \PY{o}{+} \PY{n}{theNotebook} \PY{o}{+} \PY{l+s+s1}{\PYZsq{}}\PY{l+s+s1}{\PYZcb{}}\PY{l+s+s1}{\PYZsq{}}\PY{p}{:}
                     \PY{l+s+s1}{\PYZsq{}}\PY{l+s+s1}{title\PYZob{}}\PY{l+s+s1}{\PYZsq{}} \PY{o}{+} \PY{n}{casetitle} \PY{o}{+} \PY{l+s+s1}{\PYZsq{}}\PY{l+s+s1}{\PYZcb{}}\PY{l+s+s1}{\PYZsq{}} \PY{o}{+}
                     \PY{l+s+s1}{\PYZsq{}}\PY{l+s+s1}{\PYZbs{}}\PY{l+s+s1}{subtitle\PYZob{}}\PY{l+s+s1}{\PYZsq{}} \PY{o}{+} \PY{n}{subtitle} \PY{o}{+} \PY{l+s+s1}{\PYZsq{}}\PY{l+s+s1}{\PYZcb{}}\PY{l+s+s1}{\PYZsq{}}\PY{p}{\PYZcb{}}
             \PY{n}{touchup\PYZus{}build\PYZus{}tex}\PY{p}{(}\PY{n}{theNotebook} \PY{o}{+} \PY{l+s+s1}{\PYZsq{}}\PY{l+s+s1}{.tex}\PY{l+s+s1}{\PYZsq{}}\PY{p}{,} \PY{n}{touchups}\PY{p}{)}
             \PY{c+c1}{\PYZsh{} Move the compiled pdf to the results folder}
             \PY{k+kn}{import} \PY{n+nn}{os}
             \PY{n}{publication} \PY{o}{=} \PY{n}{result\PYZus{}folder} \PY{o}{+} \PY{n}{os}\PY{o}{.}\PY{n}{sep} \PY{o}{+} \PY{n}{casename} \PY{o}{+} \PY{l+s+s1}{\PYZsq{}}\PY{l+s+s1}{.pdf}\PY{l+s+s1}{\PYZsq{}}
             \PY{k}{if} \PY{n}{os}\PY{o}{.}\PY{n}{path}\PY{o}{.}\PY{n}{exists}\PY{p}{(}\PY{n}{publication}\PY{p}{)}\PY{p}{:}
                 \PY{n}{os}\PY{o}{.}\PY{n}{remove}\PY{p}{(}\PY{n}{publication}\PY{p}{)}
             \PY{n}{os}\PY{o}{.}\PY{n}{rename}\PY{p}{(}\PY{n}{theNotebook} \PY{o}{+} \PY{l+s+s1}{\PYZsq{}}\PY{l+s+s1}{.pdf}\PY{l+s+s1}{\PYZsq{}}\PY{p}{,} \PY{n}{publication}\PY{p}{)}
\end{Verbatim}


    \begin{Verbatim}[commandchars=\\\{\}]
[NbConvertApp] Converting notebook 05b-libRadtran-dask-distributed.ipynb to latex
[NbConvertApp] Support files will be in 05b-libRadtran-dask-distributed\_files\textbackslash{}
[NbConvertApp] Making directory 05b-libRadtran-dask-distributed\_files
[NbConvertApp] Making directory 05b-libRadtran-dask-distributed\_files
[NbConvertApp] Writing 34070 bytes to 05b-libRadtran-dask-distributed.tex

    \end{Verbatim}


    % Add a bibliography block to the postdoc
    
    
    
    \end{document}
